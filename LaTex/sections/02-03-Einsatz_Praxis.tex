\subsection{Vorteile von Customer Journey Maps}
Die Vorteile von Customer Journey Maps für bestehende, aber auch für neue Produkte sind zahlreich. 
Vorzüge ergeben sich in der Nutzung von Customer Journey Maps nicht nur für Unternehmen aber auch für User.
\subsubsection{Vorteile für Unternehmen durch die Nutzung von CJMs}
Ein sehr wichtiger Punkt für Unternehmen bei der Erstellung, aber auch bei der Verbesserung von Produkten ist ein
Kundengetriebener Designprozess. Genau dieser Designprozess wird durch Customer Journey Maps vollends ausgelebt. 
Hier steht der Nutzer einer Dienstleistung wirklich im Vordergrund. Der Prozess zur Erstellung einer CJM sorgt hierbei auch
dafür, dass diese kundenzentrierte Arbeitsweise niemals in der Hintergrund gerät. Weiterhin ist die flexible Einsetzbarkeit von Customer
Journey Maps für Unternehmen sehr positiv. Mithilfe einer CJM lässt sich nicht nur der Werbeprozess für ein Produkt abbilden - Der Pre-Service Prozess -
sondern auch der tatsächliche Nutzungsprozess dieses. So lassen sich sehr langwierige Prozessketten kompakt visualisieren. 
Diese Visualisierung unterstützt im Anschluss wiederum das unternehmensinterne Verständnis für das Produkt-Ökosystem.
Dieses kann nämlich gerade bei größeren Unternehmen sehr komplex und undurchschaubar werden.
% TODO: Hier vllt Beispiel von großem Ökosystem (MS-Azure, Apple ...)
Gerade in solchen Szenarien glänzen CJMs durch ihre vorteilhafte Darstellung.
Ein weiterer Vorteil der sich bei der Nutzung von CJMs für Unternehmen auftut ist die Einbeziehung der Emotionen von Usern in den Designprozess. 
Oft wird die Verarbeitung von Emotionen in der Datenanalyse innerhalb von Unternehmen außen vor gelassen, allerdings entgleitet damit auch die Chance neue,
bisher unentdeckete Lösungswege bzw. wichtige Berührungspunkte mit dem hauseigenen Produkt zu entdecken.
Dies geht daher über die übliche quantitative Analyse der vorhandenen Daten hinaus und bietet somit, gerade auch durch die tiefgreifende Integration 
ethnografischer Methoden, die Möglichkeit die Perspektive des Users einzunehmen. 
% TODO: Vllt. Noch Vorteile?

\subsubsection{Vorteile für User durch die Nutzung von CJMs}
Auch für User ergeben sich im Kontext der Nutzung von Customer Journey Maps einige nicht zu vernachlässigende Vorteile. Gerade die starke Einflussmöglichkeit
des Users auf das zukünftige Produkt ist hier ein wichtiger Punkt. Durch den stark Nutzerzentrierten Designaspekt von Customer Journey Maps werden die Emotionen
und Ansichten von diesem zum wichtigsten Aspekt der Evaluierung des Produktes. Vorteilhaft ist dies beispielsweise, da Usern nun die Möglichkeit geboten wird
ein Produkt, welches sie sowieso schon verwenden, hin zu ihren Bedürfnissen zu verändern. Dies spart Usern die mühevolle Migration hin zu einem neuen Produkt,
welches eventuelle Bedürfnisse dann besser abbildet. Ein weiterer großer Pluspunkt, der vor allem durch die Nutzung von ethnografischen Methoden rührt, ist der,
dass auch individuelle Ansichten % TODO: (Anderes Wort)
des Users sehr gut abgebildet werden können. Subjektive Wahrnehmung, die etwa in ein User-Diary oder ein Semi-Strukturierten Interview gut integrierbar ist,
wird durch herkömmliche Datenanalyse vernachlässigt. Dies führt dazu, dass auch einzelne Meinungen von Usern nicht untergehen und zu wertvollen Ideen oder Anstößen
für Unternehmen heranwachsen können. 
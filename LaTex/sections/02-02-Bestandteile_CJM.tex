\subsection{Bestandteile einer Customer Journey Map} 
Im Folgenden werden nun die verschiedenen Bestandteile erläutert und deren Funktion und Zusammenspiel miteinander aufgezeigt. 
Bei der Definition dieser Begriffe wurde sich an der Arbeit von \cite[]{Gael2017} orientiert.
\newline
\newline
\textbf{User:} \newline
Der User ist der Hauttpteil jeder Customer Journey Map. Dieser steht bei der Erstellung und Ausweitung immer im Zentrum der Aufmerksamkeit.
User können hier alle möglichen Arten von Personen sein, die den zu modellierenden Prozess durchlaufen.
Daher kann es zu Verwirrungen führen, wie in \cite[]{Gael2017}, den User als \textit{Kunden} zu beschreiben.
Die Persona \textit{User} hat hierbei keine überwiegend kommerziellen Interessen und sollte neutral betrachtet werden.
\newline
\newline
\textbf{Journey:} \newline
Jede CJM enthält mindestens eine Journey. Diese Journey ist typischerweise eine Prozessabfolge, die von dem User befolgt wird.
Es existieren zwei Arten von Journeys. Die \textit{Expected Journey} stellt hierbei die optimale Journey dar.
Die erwartete Journey beschreibt hierbei die erwartete Strategie, die ein User bei der Prozessbewältigung wählt.
Hier werden die zu erwartenden Touchpoints festgelegt und in eine zeitliche Abfolge gebracht.
Die \textit{Actual Journey} wiederum bezeichnet die tatsächliche Journey des Users. Dies beschreibt vor allem
wie der User die Journey wahrgenommen hat und welche Probleme oder Wünsche bei der Nutzung der Diensteleistungen auftrat.
Zusätzlich ergibt sich hier die Möglichkeit, dass der User nicht genau die \textit{Expected Journey} befolgt. 
Das heißt es können beispielsweise Touchpoints übersprungen werden oder auch neue unerwartete Touchpoints hinzukommen. 
Zusätzlich besteht bei der \textit{Actual Journey} für den User immer die Möglichkeit die Reise unerwartet abzubrechen. 
\newline
\newline
\textbf{Goal:} \newline
Eine Customer Journey wird immer mit einem Ziel vor Augen modelliert. Dieses Ziel ist allerdings nicht das Ziel des Modellierenden,
sondern das des Users. Bei der Verfolgung dieses Ziels treten Interaktionen mit dem User auf und nur in Kombination mit diesem kann der Denkprozess des Users rationalisiert werden. 
\newline
\newline
\textbf{Touchpoint:}\newline
Ein Touchpoint stellt innerhalb einer CJM einen Berührungspunkt zwischen dem User und dem genutzten Produkt bzw. dem genutzten Dienst dar.
Die anordnung der Touchpoints ist hierbei arbiträr, das heißt sie folgt keinem besonderen Muster.
Touchpoints können beispielsweise auch in einem zyklischen Ablauf dargestellt werden. Darüber hinraus werden vom User meist nicht
alle dargestellten Touchpoints wahrgenommen und einige übersprungen bzw. unvorhergesehene neue Touchpoints wahrgenommen.
Zusätzlich besteht immer die Möglichkeit das der User die Journey unerwartet abbricht und somit keine weiteren Touchpoints mehr wahrnimmt. 
In \cite[]{Gael2017} wird hierbei nur von Produkten und Diensten von Unternehmen betroffen. Allerdings können auch öffentliche Prozesse
beziehungsweise nicht gewinnorientierte Abläufe Interaktionen besitzen, die mit einer CJM darstellbar sind. 
\newline
\newline
\textbf{Timeline:}\newline
Die Timeline beschreibt bei der Darstellung einer Customer Journey Map den zeitlichen Ablauf dieser Reise. 
Gerade durch die vorausschauende Herangehensweise einer \textit{Expected Journey} ist es schwierig für manche Prozesse einen sauberen
zeitlichen Ablauf darzustellen. Die Timeline stellt hierbei allerdings nicht einen distinkten Zeitpunkt eines Touchpoints dar, sondern beschreibt vielmehr den
Ablauf der verschiedenen Events, die dem User bevorstehen. 
\newline
\newline
\textbf{Experience:}\newline
Die Experience des Users beschreibt sein Feedback beziehungsweise die erlebten Emotionen beim Durchlaufen der Customer Journey. 
Gerade die Aufnahme der Emotionen in den Prozess der Evaluierung, hilft dabei den durchlaufenen Prozessablauf noch mehr auf den USer auszurichten und 
negative Emotionen oder unangenehme Situation für diesen zu vermeiden. 
\newline
\newline
\textbf{Channel:}\newline
Der Kanal ist die vom User gewählte Methode zur Interaktion mit einem Touchpoint. Beispielsweise kann ein User auf elektronischem Weg (z.B. Social Media)
oder auf traditionellem Weg (z.B. Zeitung) mit diesem interagieren. 
\newline
\newline
\textbf{Time-Stages:}\newline
Eine Customer Journey Map wird meist hinsichtlich der zeitlichen Abfolge noch in weitere Sub-Journeys unterteilt. 
Beispielsweise könnte hier eine Unterteilung in eine Pre-Service, During-Service und After-Service Stage vorgenommen werden.
Diese Stages stellen hierbei die groben zeitlichen Abfolgen in der Prozessbewältigung des Users dar. 
Dabei wird allerdings nicht nur die tatsächliche Prozessbewältigungsdauer des Users mit einbezogen, sondern auch die 
Zeiträume davor und danach. 
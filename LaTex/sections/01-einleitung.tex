\subsection{Customer Journey Maps in der heutigen Industrie}

Die heutige Industrie ist zunehmend datengetrieben und die Analyse der User mithilfe Datengesteuerter Prozesse wird immer wichtiger. In einer Studie des IW Köln im Auftrag des BDI \cite[]{BDI2021} wurden 28 Prozent
der befragten Unternehmen hinsichtlich des eigenen Datenmanagement als bereits stark digitalisiert bezeichnet. Dies zeigt, dass gerade im Bereich der Datenanalyse noch Forschritte in der Industrie gemacht werden müssen,
aber auch, dass mehr als ein viertel der Unternehmen bereits auf eine tiefgreifende Datenstrategie setzen. Gerade durch diese all Verfügbarkeit von Daten lassen sich Prozesse bzw. Handlungsschritte durch eine tiefgreifende Analyse mithilfe von Customer Journey Maps
optimieren. Dies ermöglicht es den Unternehmen Entscheidungen der User besser nachvollziehen zu können. Weiterhin lassen sich mithilfe einer Customer Journey Map eventuelle Bezugspunkte der User mit dem angebotenen Produkt besser nachvollziehen und Kontaktpunkte,
die letztendlich zum Kauf des Produktes geführt haben identifizieren. Natürlich ist auch der Einsatz von Customer Journey Maps in nicht kommerziell gesteuerten Situationen gewinnbringend. Gerade hier kann unabhängig von kapitalistischer Gewinnausrichtung eine
volle Ausrichtung auf den User gewährleistet werden. 


\subsection{Was ist eine Customer Journey Map}
 
Wie in \cite[]{Marquez2015} beschrieben ist eine Customer Journey Map (CJM) visuelle Darstellung der User Journey und der Erfahrung bei der Nutzung eines Dienstes oder Bereichs.
Die CJM visualisiert die User Journey von Anfang bis Ende einer Aufgabe, um die verschiedenen Phasen, Schritte und Touchpoints, die ein Benutzer durchlaufen muss, um eine Aufgabe zu erledigen, hervorzuheben und zu verstehen.
Ziel hierbei ist es den zu durchlaufenden Prozess aus der Sicht des Users zu betrachten. Der Fokus bei der Erstellung einer CJM ist also deutlich auf die die Emotionen des Users in hinsicht seiner User Journey gesetzt.
Wendet man diese User-zentrierte Technik konsequent an so erhält man eine Analyse eines Prozesses, welche die Möglichkeit bietet die Reise des Users deutlich zu verbessen.
Das Ziel bei der Erstellung beziehungsweise der Anwendung einer Customer Journey Map kann sich natürlich auch mit den Motiven der Ersteller ändern. 
Während es das Ziel einer Customer Journey Map in Reinform ist die User Experience möglichst optimal zu gestalten ergeben sich in der freien Wirtschaft oft Abstriche. 
Jedoch ergeben sich auch im gewinnfokusierten Einsatz von CJMs Vorteile. Ein wichtiger Aspekt ist hierbei die Generierung von Erkenntnissen über das Verhalten und die Präferenzen der Zielgruppe im Bezug auf Berührungspunkte mit dem Produkt zum Beispiel durch Werbung.
Dadurch kann festgestellt werden welche dieser Punkte besonders ausschlaggebend für den Kauf des Produktes waren. 
Weiterhin ergibt sich die Möglichkeit die Effizienz verschiedener Kanäle zu analysieren und vor allem auch die Wirkung dieser Kanäle aufeinander zu obersvieren.  

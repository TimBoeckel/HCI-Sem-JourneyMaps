\subsection{Verstehen der User-Agenda - [Sergej Gelver]}
Der Aufbau einer Customer Journey Map ist, trotz der Zahlreichen Variationen in der Menge der Phasen, der einzelnen Stufen oder der Grafischen Repräsentation,
Grundsätzlich gleich aufgebaut. Jede Customer Journey Map hat einen Ziel, das ein Kunde zu erreichen versucht und dafür mehrere Phasen, Touchpoints und Channels durchläuft,
während es das Ziel des Unternehmens ist, diese Berührpunkte so zu gestalten, dass Sie keinen negativen, bzw. im Bestfall einen positiven, Einfluss auf den Kunden haben.
\newline
\newline
\subsubsection{Prozess der Erstellung [Sergej Gelver]}
Eine Customer Journey Map wird in der Regel von 2 bis 6 Personen in einem mehrstündigen Workshop erstellt. Hierbei wählt man einen Prozess den ein Kunde Potentiell durchlaufen
soll. Es werden nun Touchpoints entdeckt und definiert und in einer Chronologischen Reihenfolge an einem Whiteboard bzw. auf dem Tisch gelegt und weiterhin wird sich Gedanken gemacht
in welcher Form ein Kunde mit diesen Touchpoints interagiert, sogenannte Kanäle. Durch vorher Recherchierte Informationen können Customer Journey Maps noch Konkreter und 
Detailierter werden indem man weitere Ebenen hinzufügt. Solche Ebenen unterscheiden sich je nach definiertem Ziel der CJM und dem Produkt des Unternehmens.
\newline
\newline
\subsubsection{Aufbau einer Customer Journey Map [Sergej Gelver]}
Sowohl die Ebenen als auch die Phasen einer Customer Journey Map sind nicht festgelegt auf Menge, Art oder Name. So kann eine Customer Journey Map aus Drei,
Fünf, Acht oder sogar mehr Phasen bestehen und die Unterschiedlichsten Merkmale und Charakteristika besitzen. Die Gemeinsamkeit die die meisten CJM jedoch teilen ist die, 
das Phasen in der Regel in drei Überkategorien eingereiht werden können, nämlich die Pre-Service, Service und Post-Service "Phasen". Je mehr Phasen ein Unternehmen also 
für seine CJM bestimmt, desto Präziser sind die einzelnen Schritte des Kunden durchdacht.
\newline
 Im folgenden betrachten wir die Darstellung der Phasen wie sie in der Arbeit von \cite[]{Ozarslan2015} beschrieben werden.
 \newline
 \newline
\textbf{Problem Recognition} \newline
In dieser Phase verspürt der Nutzer ein Bedürfnis bzw. erkennt ein Problem und will dieses Lösen. Der Auslöser für die Problem Erkennung kann sowohl ein Interner als auch ein Externer sein.
Beispiele für Interne Auslöser können natürliche Empfindungen wie Hunger, Müdigkeit oder Schmerz sein. ein Externer Auslöser hingegen, ist ein äußerlicher Einfluss, wie zum Beispiel ein Werbeplakat,
welches für ein Produkt wirbt das man gerne hätte.
\newline
\newline
\textbf{Information Search} \newline
In dieser Phase Informiert sich der User über Möglichkeiten oder Alternativen. Diese Informationssuche funktioniert entweder über seinen bereits vorhandenes wissen, wie beispielsweise
andere Hersteller des selben Produkts, oder Recherche durch andere Quellen wie z.B. das Internet.
\newline
\newline
\textbf{Evaluation of Alternatives} \newline
Der Kunde vergleicht nun seine Alternativen, um in dieser Phase eine Entscheidung zu treffen. Um sich jedoch entscheiden zu können braucht es 4 Schritte. Zuerst müssen die 
Bewertungskriterien für den Erwerb festgelegt werden. Im zweiten Schritt werden alle Alternativen Identifiziert. In Schritt Drei werden anhand der in Schritt Eins festgelegten Bewertungskriterien
die Leistung der Alternativen verglichen. Letztlich wird im vierten Schritt Entscheidungsregeln angewendet. 
\newline
\newline
\textbf{Purchase} \newline
Nachdem in Phase 3 eine Entscheidung getroffen wurde welches Produkt von welchem Hersteller gekauft werden soll, wird der Kauf in dieser Phase abgewickelt. Üblicherweise hat
das erworbene Produkt die beste Leistung, ausgehend von den Bewertungskriterien aus der vorherigen Phase. Damit ist nicht zwingend Leistung im herkömmlichen Sinne gemeint, da
der Preis ebenfalls ein gängiges Bewertungskriterium ist.
\newline
\newline
\textbf{Post-Purchase Evaluation} \newline
Finale Phase in der ein Kunde letztlich Evaluiert, ob er seinen Einkauf bereut oder mit dem Produkt zufrieden ist. 
\newline
\newline
\textbf{weitere gemeinsamkeiten} \newline
In der Grafischen Repräsentation sind Touchpoints innerhalb der Einzelnen Phasen dargestellt. Dadurch lässt sich nicht nur ein Chronologischer Ablauf erkennen und die somit vorgesehen
"Journey" des Kunden, sondern dient auch als Bewertungssystem. Die durch Interviews (dazu mehr in 2.3 Methoden) gewonnen Informationen werden in einer 1 bis 5 punkte Skala verteilt und 
stellen die Zufriedenheit des Kunden dar, der mit diesem Touchpoint interagiert hat. Solch eine Bewertung ist Relevant weil in vielen fällen schon ein negativer Touchpoint 
genügt, um einen Kunden von weiter Interaktion mit dem Unternehmen abzuhalten. Somit weis das Unternehmen welche Risiken in welche Phase existieren und wie diese anzugehen sind.
\newline
\newline
\begin{figure}[H]
    \includegraphics[scale=0.5]{ image/02-01-Verstehen_der_Usergenda/CJM.PNG}
    \caption{Graphische Darstellung der 5 Phasen einer CJM \cite[]{Ozarslan2015}}
\end{figure}
des Weiteren sind Kanäle die wichtigste und häufigste Eigenschaft einer CJM nach Touchpoints. Kanäle stellen einen direkten Bezug dazu auf, wie ein Kunde mit einem Touchpoint, und somit
mit dem Unternehmen Interagiert. So wäre z.B. Werbung ein Touchpoint, während der "Social Media" der Kanal wäre. Kanäle haben ebenfalls Starken Einfluss darauf, wie ein Touchpoint wahrgenommen
wird und kann die Interaktion für einen Kunden angenehmer Gestalten.
\subsubsection{}
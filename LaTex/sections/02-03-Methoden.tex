\subsection{Methoden zur Erstellung und Verbesserung einer Customer Journey Map}
Bei der Erstellung von Customer Journey Maps werden vor allem Methoden aus der ethnographischen Forschung genutzt. 
Dies ermöglicht die intensive auseinandersetzung mit der Lebenswelt des jeweiligen Users.
Der Wert der ethnografischen Forschung liegt dabei in  Fähigkeit,
reichhaltige Daten über individuelle Userinteraktionen zu liefern. Ethnografie ist für die explorative Forschung förderlich,
da sie unbekannte Probleme und Einblicke in das Userverhalten aufdecken kann. Im Folgenden werden nun einige dieser
ethnografischen und auch nicht-ethnografischen Methoden beschrieben und deren Anwendung im Kontext von Customer Journey Maps verdeutlicht.
Natürlich kann nicht jede der vorgestellten Methoden gleich gut auf jeden Prozess angewendet werden. Hier spielen Faktoren wie zum Beispiel die Länge 
der zu analysierenden User Journey eine wichtige Rolle. Gerade deshalb ist es wichtig vor der Erstellung einer CJM tiefgreifende Überlegungen
für das Auswählen eines geeigneten Methodensets zu machen. 



\subsubsection{Interviews}
Bei der Erstellung von Customer Journey Maps werden meist Interviews genutzt, welche entweder Semi- oder unstrukturiert sind. 
Wie in \cite[]{Bauman1992} beschrieben enthält  ein unstrukturiertes Interview im Allgemeinen keine vordefinierten Fragen oder Themen und gibt keine Reihenfolge des Informationsflusses vor.
Die Befragten werden ermutigt, über ein vom Forscher ausgewähltes Thema zu sprechen,
 aber die spezifischen Unterthemen, Subbereiche und die Reihenfolge der Diskussion werden von den Befragten und ihren von Prioritäten bestimmt.
Das Interview ist lediglich ein Rahmen, in dem die Befragten ihre eigenen Auffassungen in ihren eigenen Begriffen ausdrücken können.
Dies hilft vor allem bei der Erstellung einer Customer Journey Map dabei unbekannte Touchpoints zu identifizieren und diese in die CJM zu integrieren.
Zusätzlich lassen sich aus einem solchen Interview natürlich auch die Erfahrung bzw. die Emotionen des Users ablesen. Dies wiederrum trägt zu einer 
möglichen Gewichtung der Touchpoints bei.
%TODO Mehr auf CJM

\subsubsection{Focus Groups}
Wie in \cite[]{Kitzinger1995} beschrieben sind Focus Groups eine Form des Gruppeninterviews, die die Kommunikation zwischen den Forschungsteilnehmern ausnutzt um Daten zu generieren.
Obwohl Gruppen Gruppeninterviews häufig einfach als schnelle und und bequeme Art gesehen werden, Daten von mehreren Personen gleichzeitig zu  erheben, nutzen Fokusgruppen Interaktion als Teil der Methode.
Dies bedeutet, dass der Forscher nicht jede Person bittet, der Reihe nach Fragen zu beantworten sondern die Personen ermutigt auch
Die Idee hinter der Fokusgruppenmethode ist es, dass Gruppenprozesse den Menschen helfen können, ihre Ansichten zu erforschen 
und diese in so zu artikulieren, wie es in einem Einzelinterview nicht so leicht möglich wäre. 
Im Bezug auf Customer Journey Maps ergibt sich hier die Möglichkeit weitere Informationen über einen Prozess zu gewinnen. Besonders wichtig ist dabei jedoch,
dass in einem solchen Gruppeninterview auch zwischenmenschliche Interaktionen während der User Journey mit aufgenommen werden können. 
Dies kann unter Umständen dazu beitragen auch prozessbezogene Berührungspunkte zwischen Menschen zu identifizieren und daraufhin in die Customer Journey Map zu integrieren. 
%TODO CJM ausführen

\subsubsection{User Diary}
Die Methode des User Diarys ist ein Selbstberichtsinstrument und wurde in vielen Studien zur Untersuchung laufender
Erfahrungen eingesetzt. Wie in \cite[]{Bolger2003} beschrieben bietet ein User Diary die Möglichkeit, soziale, psychologische und sogar physiologische
Prozesse innerhalb von Alltagssituationen zu untersuchen. User-Diarys unterscheiden sich von anderen sozialwissenschaftlichen Methoden dadurch,
 dass die Zeit zwischen dem Auftreten eines Ereignisses und der Aufzeichnung der Informationen verkürzt wird. Dies hat den Vorteile das eine solche Methode
weniger anfällig für Erinnerungslücken ist, wie es bei retrospektiven Interviews der Fall sein kann.
Außerdem kann, wie in \cite[]{Rieman1993} beschrieben, ein User-Diary dabei helfen, die Lücke zwischen den naturalistischen Untersuchungsmethoden (z.B. Beobachtungen im Feld) und den
(z.B. Beobachtungen im Feld) und den Labormethoden (d.h. Untersuchung von Usern in einem kontrollierten Labor) zu überbrücken, indem sie
indem sie objektive Daten über den realen Kontext von Menschen im Feld liefern, die weitere Untersuchungen im Labor anleiten können.
Zusätzlich ergreifen die User die Initiative, um Informationen zu erfassen, was eine Datenerfassung in Echtzeit ermöglicht.
Gerade im Bezug auf Customer Journey Maps ist die Anwendung von User Diarys für länger Andauernde Prozesse zu empfehlen, da ansonsten die Anwendung anderer Methoden einfacher und gewinnbringender für die Untersuchung ist. 
Wenn sich die Anwendung aber auf einen Customer Journey Mapping Prozess passend anwenden lässt, erhält man hier die Möglichkeit reichhaltige und strukturierte Informationen zu identifikation von Touchpoints
beziehungsweise zu deren Gewichtung bekommen. 

\subsubsection{Field Visits and User Observation}
In einer Feldstudie werden die User direkt vor Ort in ihren normalen Umfeld beobachtet. 
Wie in \cite[]{Goodman2012} Observation des Prozesses werden Informationen über die Umgebung, in der die Menschen agieren generiert,
die sonst nicht für Forschende zugänglich wären. Die Feldstudie hilft, die Aktionen der User im Kontext dieses Umfelds zu interpretieren.
Indem die Welt an der Seite der User erlebt wird, kann besser nachvollzogen werden, welche Probleme die User im Prozess haben und vor welche Herausforderungen sie gestellt werden. Diese grundlegende Forschungsmethode beinhaltet,
dass die User einmal oder mehrmals besucht werden, ihnen Fragen gestellt werden und sie bei ihren normalen Aktivitäten begleiten werden.
Durch die Anwendung dieser Methode zur Erstellung einer Customer Journey Map, können hier auch Probleme oberviert werden, die von den Usern schwierig in Sprache artikuliert werden.
Zusätzlich können Umstände, die von den Usern als normal und damit nicht erwähnenswert interpretiert werden, von Forschern analysiert werden und zu eventuellen Touchpoints in 
der Customer Journey Map werden.


\subsubsection{Hierarchical Task Analysis}
Die Erstellungsprozess der Hierarchical Task Analysis besteht darin, Aufgaben in Teilaufgaben mit beliebigem Detaillierungsgrad zu zerlegen.
Jede Teilaufgabe oder Operation wird durch ein Ziel, die Eingabebedingungen,
 unter denen das Ziel aktiviert wird, die zum Erreichen des Ziels erforderlichen Aktionen und das Feedback, das die Zielerreichung anzeigt, spezifiziert.
Die Beziehung zwischen einer Menge von Teilaufgaben und der übergeordneten Aufgabe wird als Plan bezeichnet,
 und Es können mehrere Planarten unterschieden werden, darunter Prozeduren, selektive Regeln und Time-Sharing Aufgaben.
  Das übergeordnete Ziel der Analyse ist, analog zum Einsatz in Customer Journey Maps, die Identifizierung tatsächlicher oder möglicher Ursachen für Probleme
und geeignete Abhilfemaßnahmen vorzuschlagen, wie z. B. die Modifikation des Aufgabendesigns.
Die Hierarchical Task Analysis hilft dabei auch in der Anwendung bei der Erstellung einer CJM noch unbekannte Subtasks des Prozesses zu erforschen und damit mögliche Touchpoints
mit dem User zu finden. Vorteilhaft ist hierbei, dass man für die Anwendung einer solchen Taskanalyse nicht zwingend einen User vor Ort braucht. 
Das heißt die HTM ist unabhängiger als einige der anderen vorgestellten Methoden. Dies kann besonders bei der Erstellung einer \textit{Expected Journey} sehr hilfreich sein und
die Möglichkeit bieten ein strukturiertes Vorgehen zu forcieren. 
% TODO Mehr auf CJM eingehen
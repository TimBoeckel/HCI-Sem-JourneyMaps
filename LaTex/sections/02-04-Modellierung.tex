\subsection{Modellierung einer Customer Journey Map [Tim Boeckel]}
%TODO Viele Repräsentationsmglkt.
Es gibt viele verschiedene Möglichkeiten Customer Journey Maps zu modellieren. 
Besonders gut geeignet sind hierfür Diagramformen, welche eine simple visuelle Darstellung der Komponenten einer CJM fördern.
Gerade durch die besonderen visuellen Eigenschaften einer CJM, sollte man durch die Visualisierung in der Lage sein komplexe
Handlungsabläufe des Users kompakt darzustellen und somit Verständiss für dieses zu fördern.
Im Folgenden werden zwei dieser vielen Möglichkeiten zur Visualisierung und Modellierung einer Customer Journey Map dargestellt. 

\subsubsection{Sequenz-Diagramm [Tim Boeckel]}
Eine Darstellungsweise für eine Customer Journey durch ein Diagram ist die Visualisierung durch ein Sequenz-Diagram. 
Wie in \cite[]{Haugstveit2016} beschrieben lässt sich mithilfe dieser Art von Diagram zuerst die optimale Journey modellieren. 
Dies zeigt wie in ~\ref{fig:seqOpt} dargestellt die "optimale" Journey. Hier werden Touchpoints und deren Beziehung zueinander 
definiert. Kanäle können hier beispielsweise, wie in ~\ref{fig:seqChan} dargestellt, durch Icons innerhalb der Touchpoints symbolisiert werden.
Natürlich können Touchpoints auch in Zyklen auftreten und verschiedene Abzweigmöglichkeiten geben.
\begin{figure}[H]
    \includegraphics{image/02-04-Modellierung/Sequenzdiagram_optimal.PNG}
    \caption{Model Darstellung einer optimalen User Journey durch ein Sequenz-Diagram \cite[]{Haugstveit2016}}
    \label{fig:seqOpt}
\end{figure}

\begin{figure}[H]
    \includegraphics[scale=0.8]{image/02-04-Modellierung/Sequenzdiagram_channels.PNG}
    \caption{Darstellung von kanalbasierten Touchpoints in einem Sequenz-Diagram aus \cite[]{Haugstveit2016}}
    \label{fig:seqChan}
\end{figure}

Allerdings stellt die optimale Customer Journey sehr selten die tatsächliche Reise dar. In der tatsächlichen Reise können beispielsweise Touchpoints
übersprungen werden oder aber andere nicht eingeplante Touchpoints besucht werden. Wie in  \figurename{~\ref{fig:seqDev}} dargestellt können
mithilfe eines Sequenz-Diagrams solche Abweichungen nach einer Evaluierung der tatsächlichen Customer Journey inkludiert werden. 
Dies bietet die Möglichkeit durch das Übereinanderlegen der optimalen und tatsächlichen Customer Journey eine Neuevaluierung des Produktes zu starten.

\begin{figure}[H]
    \includegraphics[width=15cm]{image/02-04-Modellierung/Sequenzdiagram_deviation.PNG}
    \caption{Model Darstellung einer abweichenden User Journey durch ein Sequenz-Diagram aus \cite[]{Haugstveit2016}}
    \label{fig:seqDev}
\end{figure}

\subsubsection{Swimlane-Diagramm [Tim Boeckel]}
Eine weitere Modellierungsform, die zur Visualisierung einer Customer Journey Map genutzt werden kann, ist das Swimlane Diagram. Wie in \figurename{~\ref{fig:swim}} dargestellt können hier,
ähnlich wie in einem Sequenz-Diagram, Touchpoints in einem zeitlichen Ablauf dargestellt werden. Besonders interessant ist an dieser Darstellungsform jedoch die Möglichkeit auch
Touchpoints zwischen Usern zu modellieren. Dies bietet die Möglichkeit vorher unbekannte Touchpoints zu modellieren und auch mehrere Akteure in die Darstellung der CJM mit einzubeziehen. 


\begin{figure}[H]
    \includegraphics[width=15cm]{image/02-04-Modellierung/Swimlanediagram.PNG}
    \caption{Model Darstellung einer abweichenden User Journey durch ein Swimlane-Diagram aus \cite[]{Haugstveit2016}}
    \label{fig:swim}
\end{figure}
\subsection{Modellierung einer Customer Journey Map}
%TODO Viele Repräsentationsmglkt.
Es gibt viele verschiedene Möglichkeiten Customer Journey Maps zu modellieren. 
Besonders gut geeignet sind hierfür Diagrammformen, welche eine simple visuelle Darstellung der Komponenten einer CJM fördern.
Gerade durch die besonderen visuellen Eigenschaften einer CJM, sollte man durch die Visualisierung in der Lage sein komplexe
Handlungsabläufe des Users kompakt darzustellen und somit Verständiss für dieses zu fördern.

\subsubsection{Sequenz-Diagramm}
Eine sehr populäre Darstellungsweise für eine Customer Journey durch ein Diagramm ist die Visualisierung durch ein Sequenz-Diagram. 
Wie in \cite[]{Haugstveit2016} beschrieben lässt sich mithilfe dieser Art von Diagramm zuerst die optimale Journey modellieren. 
Dies zeigt wie in ~\ref{fig:seqOpt} dargestellt die "optimale" Journey. Hier werden Touchpoints und deren Beziehung zueinander 
definiert. Kanäle können hier beispielsweise, wie in ~\ref{fig:seqChan} dargestellt, durch Icons innerhalb der Touchpoints symbolisiert werden.
Natürlich können Touchpoints auch in Zyklen auftreten und verschiedene Abzweigmöglichkeiten geben.
\begin{figure}[h]
    \includegraphics{image/02-04-Modellierung/Sequenzdiagram_optimal.PNG}
    \caption{Model Darstellung einer optimalen User Journey durch ein Sequenz-Diagramm}
    \label{fig:seqOpt}
\end{figure}

\begin{figure}[h]
    \includegraphics[scale=0.8]{image/02-04-Modellierung/Sequenzdiagram_channels.PNG}
    \caption{Darstellung von kanalbasierten Touchpoints in einem Sequenz-Diagramm}
    \label{fig:seqChan}
\end{figure}

Allerdings stellt die optimale Customer Journey sehr selten die tatsächliche Reise dar. In der tatsächlichen Reise können beispielsweise Touchpoints
übersprungen werden oder aber andere nicht eingeplante Touchpoints besucht werden. Wie in  \figurename{~\ref{fig:seqDev}} dargestellt können
mithilfe eines Sequenz-Diagramms solche Abweichungen nach einer Evaluierung der tatsächlichen Customer Journey inkludiert werden. 
Dies bietet die Möglichkeit durch das Übereinanderlegen der optimalen und tatsächlichen Customer Journey eine Neuevaluierung des Produktes zu starten.

\begin{figure}[h]
    \includegraphics[width=15cm]{image/02-04-Modellierung/Sequenzdiagram_deviation.PNG}
    \caption{Model Darstellung einer abweichenden User Journey durch ein Sequenz-Diagramm}
    \label{fig:seqDev}
\end{figure}

\subsubsection{Swimlane-Diagramm}


\begin{figure}[h]
    \includegraphics[width=15cm]{image/02-04-Modellierung/Swimlanediagram.PNG}
    \caption{Model Darstellung einer abweichenden User Journey durch ein Swimlane-Diagramm}
    \label{fig:swim}
\end{figure}
\subsection{Methoden zur Erstellung einer Customer Journey Map}
Bei der Erstellung von Customer Journey Maps werden vor allem Methoden aus der ethnographischen Forschung genutzt. 
Dies ermöglicht die intensive auseinandersetzung mit der Lebenswelt des jeweiligen Users.
Der Wert der ethnografischen Forschung liegt dabei in  Fähigkeit,
reichhaltige Daten über individuelle Kundeninteraktionen zu liefern. Ethnografie ist für die explorative Forschung förderlich,
da sie unbekannte Probleme und Einblicke in das Kundenverhalten aufdecken kann. Im Folgenden werden nun einige dieser
ethnografischen Methoden beschrieben und deren Anwendung im Kontext von Customer Journey Maps verdeutlicht.

\subsubsection{User Diary}
Die Führung eines User-Tagebuchs

\subsubsection{Interviews}

\subsubsection{Hierarchical Task Analysis}
\subsection{Verstehen der User-agenda}
Der Aufbau einer Customer Journey Map ist, trotz der Zahlreichen Variationen in der Menge der Phasen, der einzelnen Stufen oder der Grafischen Repräsentation,
Grundsätzlich gleich aufgebaut. Jede Customer Journey Map hat einen Ziel, das ein Kunde zu erreichen versucht und dafür mehrere Phasen, Touchpoints ound Channels durchläuft,
während es das Ziel des Unternehmens ist, diese Berührpunkte so zu gestalten, dass Sie keinen negativen, bzw im bestfall einen positiven, Einfluss auf den Kunden haben.
\newline
\newline
\subsubsection{Prozess der Erstellung}
Eine Customer Journey Map wird in der Regel von 2 bis 6 Personen in einem mehrstündigen Workshop erstellt. Hierbei wählt man einen Prozess den ein Kunde Potentiell durchlaufen
soll. Es werden nun Touchpoints entdeckt und definiert und in einer Chronologischen reihenfolge an einem Whiteboard bzw auf dem Tisch gelegt und weiterhin wird sich gedanken gemacht
in welcher Form ein Kunde mit diesen Touchpoints interagiert, sogenannte Kanäle. Durch vorher Recherchierte Informationen können Customer Journey Maps noch Konkreter und 
Detailierter werden indem man weitere Ebenen hinzufügt. Solche Ebenen unterscheiden sich je nach definiertem Ziel der CJM und dem Produkt des Unternehmens.
\newline
\newline
\subsubsection{Aufbau einer Customer Journey Map}
Sowohl die Ebenen als auch die Phasen einer Customer Journey Map sind nicht festgelegt auf Menge, art oder Name. So kann eine Customer Journey Map aus Drei,
Fünf, Acht oder sogar mehr Phasen bestehen und die Unterschiedlichsten Merkmale und Charaktaristika besitzen. Die Gemeinsamkeit die die meisten CJM jedoch teilen ist die, 
das Phasen in der Regel in drei Überkategorien eingereiht werden können, nämlich die Pre-service, Service und Post-service "Phasen". Je mehr Phasen ein Unternehmen also 
für seine CJM bestimmt, desto Präziser sind die einzelnen Schritte des Kunden durchdacht.
\newline
 Im folgenden betrachten wir eine CJM mit Fünf Phasen, weil das die am weitesten verbreitete Darstellung ist.
 \newline
 \newline
\textbf{Phase 1} \newline
Auch in der Namensgebung Variieren die einzelnen Phasen. "Awareness", "Attention" oder "Consider" wird die erste Phase oft genannt. In dieser Phase hat der Kunde
ein bedürfnis welches er zu befriedigen versucht. Durch sinnvoll platzierte werbung, gute Internetpräsenz oder Social Media, soll er 
auf das Unternehmen aufmerksam gemacht werden, welches sein Bedürfnisse befriedigen kann. \textbf{TO BE CONTINUED!!!}
\newline
\newline
\textbf{Phase 2} \newline
"Consideration" oder "Explore" sind häufig verwendete bezeichnungen für diese Phase. In dieser Phase hat der Kunde bereits eine ersten Kontakt
mit dem Unternehmen gehabt, er weis über die existenz und das Produkt und soll von dem Angebot überzeugt werden. \textbf{TO BE CONTINUED!!!}
\newline
\newline
\textbf{Phase 3} \newline
"Purchase" oder Acquisition", ist die Phase, in der Kunde erfolgreich von dem Angebot eines Unternehmens überzeugt wurde und das Produkt bzw. die Dienstleistung kauft.
\newline
\newline
\textbf{TO BE CONTINUED!!!}
\newline
\newline
\textbf{Phase 4} \newline
Nachdem erwerb des Produktes oder der Dienstleistung befindet sich der Kunde in der Phase "Retention"oder "Service" in welcher der Kunde seinen Kauf ausgehändigt, oder geliefert bekommt
und das Hauptziel des Kunden erreicht wird.\textbf{TO BE CONTINUED!!!}
\newline
\newline
\textbf{Phase 5} \newline
Zuletzt befindet sich der Kunde in der letzten, der "Loyalty Expansion" oder auch "Advocacy" Phase, in welcher das Hauptgeschäft abgeschlossen ist, das Unternehmen jedoch
weiter im Kontakt bleibt um für einen eventuellen weiteren einkauf Attraktik zu bleiben. \textbf{TO BE CONTINUED!!!}
\newline
\newline
\textbf{weitere gemeinsamkeiten} \newline
In der Grafischen Repräsentation sind Touchpoints innerhalb der Einzelnen Phasen dargestellt. Dadurch lässt sich nicht nur ein Chronologischer Ablauf erkennen und die somit vorgesehen
"journey" des Kunden, sondern dient auch als Bewertungssystem. Die durch Interviews (dazu mehr in 2.3 Methoden) gewonnen Informationen werden in einer 1 bis 5 punkte Skala verteilt und 
stellen die Zufriedenheit des Kunden dar, der mit diesem Touchpoint interagiert hat. Solch eine Bewertung ist Relevant weil in vielen fällen schon ein negativer Touchpoint 
genügt, um einen Kunden von weiter Interaktion mit dem Unternehmen abzuhalten. Somit weis das Unternehmen welche Risiken in welche Phase existieren und wie diese anzugehen sind.
\newline
\newline
des Weiteren sind Kanäle die wichtigste und häufigste Eigenschaft einer CJM nach Touchpoints. Kanäle stellen einen direkten Bezug dazu auf, wie ein Kunde mit einem Touchpoint, und somit
mit dem Unternehmen Interagiert. So wäre z.B. Werbung ein Touchpoint, während der "Social Media" der Kanal wäre. Kanäle haben ebenfalls Starken einfluss darauf, wie ein Touchpoint wahrgenommen
wird und kann die Interaktion für einen Kunden angenehmer Gestalten.
\subsubsection{}
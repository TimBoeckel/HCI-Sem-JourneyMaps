\subsection{Bestandteile einer Customer Journey Map} 

\textbf{Journey:} \newline
Jede CJM enthält mindestens eine Journey. Diese Journey ist typischerweise eine Prozessabfolge, die von dem User befolgt wird.
Es existieren zwei Arten von Journeys. Die \textit{Expected Journey} stellt hierbei die optimale Journey dar.
Diese wird beispielsweise genutzt, um Möglichkeiten für neue Features zu explorieren. 
Die \textit{Actual Journey} wiederum bezeichnet die tatsächliche Journey des Users. Dies beschreibt vor allem
wie der User die Journey wahrgenommen hat und welche Probleme oder Wünsche bei der Nutzung der Diensteleistungen auftrat. 
\newline
\newline
\textbf{Goal:} \newline
Eine Customer Journey sollte mit einem Ziel vor Augen abgebildet werden,
das auch als Szenario, Story oder Hauptintention bezeichnet wird. Das Ziel löst Interaktionen mit Nutzern
aus und stellt den Denkprozess der Benutzer dar.
\newline
\newline
\textbf{Touchpoint:}\newline
Ein Touchpoint ist eine Interaktion zwischen Kunden und Produkten oder Dienstleistungen von Unternehmen
Produkten oder Dienstleistungen, wie z. B. "Suche nach einem Produkt", oder
"Finden von Sitzplätzen". Die Anordnung der Touchpoints kann zyklisch sein: ein Kunde
kann mehrmals über dieselben Touchpoints iterieren. Außerdem kann die Anordnung
nicht-linear sein: 
\begin{enumerate}
    \item Der Kunde durchläuft meist nicht alle vorhandenen Touchpoints
    \item Der Kunde kann einen geplanten Touchpoint verpassen
    \item Der Kunde kann die Reise unerwartet abbrechen
\end{enumerate}
\textbf{Timeline:}\newline
Die Timeline beschreibt die Dauer der Reise vom ersten
bis zum letzten Touchpoint. Aufgrund des unswicheren Charakters der zu erwartenden Journey
wäre es nicht verwunderlich, wenn keine Zeitindikation vorhanden wäre. Dennoch kann eine Zahl, die einem
einem Ereignis (d. h. einem Touchpoint) kann jedoch die Reihenfolge innerhalb der Zeitleiste dargestellt werden.
\newline
\newline
\textbf{Experience:}
Die Experience umfasst das Feedback und die Emotionen der Kunden. Sie beschreibt die Erfahrung mit dem Produkt 
durch die Angabe der Gefühle hinsichtlich der Journey bzw. des Produktes.
\textbf{Channel:}\newline
Der Kanal ist die vom Kunden gewählte Methode zur Interaktion
mit dem Touchpoint, z. B. "Soziale Medien".


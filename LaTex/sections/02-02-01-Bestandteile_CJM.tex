\subsection{Bestandteile einer Customer Journey Map} 

\textbf{Journey:} \newline
Jede CJM enthält mindestens eine Journey. Diese Journey ist typischerweise eine Prozessabfolge, die von dem User befolgt wird.
Es existieren zwei Arten von Journeys. Die \textit{Expected Journey} stellt hierbei die optimale Journey dar.
Diese wird beispielsweise genutzt, um Möglichkeiten für neue Features zu explorieren. 
Die \textit{Actual Journey} wiederum bezeichnet die tatsächliche Journey des Users. Dies beschreibt vor allem
wie der User die Journey wahrgenommen hat und welche Probleme oder Wünsche bei der Nutzung der Diensteleistungen auftrat. 
\newline
\textbf{Goal:} \newline
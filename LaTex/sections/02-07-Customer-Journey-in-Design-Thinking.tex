\subsection {Design Thinking und Customer Journey Maps [Sergej Gelver]} 
Customer Journey Maps größte Stärke ist das Tiefe und gründliche Verständnis für seinen Endverbraucher. Dem Kunden wird die Möglichkeit geboten
indirekt an der Weiterentwicklung eines Produktes mitzuwirken und gleichzeitig dafür Wertschätzung zu erfahren. 
\newline
\newline
Was eine Customer Journey jedoch nicht eigenständig kann, beziehungsweise nur in begrenztem Maße, ist es, ein Produkt von Grund auf zu Entwickeln.
Darin liegt ganz klar die Stärke von Design Thinking. Hat man ein Produkt auf dem man aufbauen kann, so ist es möglich entweder Mithilfe von Design Thinking einen innovativen Ansatz
zu wählen um das Produkt neu zu erfinden oder zu verbessern, oder man Entscheidet sich für den weg der Customer Journey, bezieht alle notwendigen Daten von Kunden ein und passt das 
Produkt entsprechend den Wünschen der Kunden an.
\newline
\newline
Jedoch ist Das keine entweder/oder Entscheidung. Customer Journey Maps und Design Thinking stehen nicht in Direkter Konkurrenz. Es ist durchaus gängige Praxis
beide Prozesse in Symbiose zu verwenden. So spielt der Prozess der CJM gerade in der ersten Phase, "Verstehen" von Design Thinking eine große Rolle und ist 
ein beliebtes Werkzeug um Unternehmen mitzuteilen welche Bedürfnisse ihre Kunden haben und wie man diese nach Möglichkeit befriedigt.
\newline
\newline
Allerdings werden nicht immer alle aus CJM's gewonnen Informationen verwendet, um ein Produkt zu Weiterzuentwickeln. So kann es aus Diversen Gründen sogar soweit kommen,
das Aktiv gegen die Wünsche der Kunden gehandelt wird. Insbesondere Apple ist ein Starker Vertreter dieses Verhaltens, wie sie in Zahlreichen Beispielen und Entscheidungen bewiesen haben.

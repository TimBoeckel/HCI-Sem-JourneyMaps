\subsection{User und Customer - Sergej Gelver}
Der Unterschied der Begrifflichkeiten ist auf den ersten Blick eindeutig. Ein Customer kauft ein Produkt, dabei ist jedoch nicht klar, ob er es verwendet oder es für jemand anderes kauft.
Ein User hingegen benutzt ein Produkt und ist nicht zwangsläufig der Kunde, der das Produkt bezahlt hat.
\newline
\newline
Es stellt sich also die Frage ob ein Unternehmen oder eine Institution diese beiden Personengruppen differenziert betrachten sollte, wenn es darum geht seinen Service oder sein Produkt
so Attraktiv wie möglich zu gestalten. Um einen vergleich möglich zu machen, abseits der Wort Definition, betrachten wir die zwei Prozesse die beide Personengruppen in den Mittelpunkt
stellen. Für den Customer ist es die Customer Journey, während wir für den User auf die User Experience schauen.
\newline
\newline
\textbf{Customer Journey Maps}
\newline
Das Ziel einer Customer Journey ist es, die Interaktion des Kunden mit dem eigenen Unternehmen oder der Institution so angenehm und reibungslos wie möglich zu gestallten, indem
man sich bei Planung und Ausarbeitung spezieller Touchpoints in dessen Rolle versetzt. Insbesondere wird darauf geachtet, wie wohl sich ein Kunde zu welchem Zeitpunkt und in welcher Situation
fühlt, um festzustellen wie woran das liegt und wie man entgegen wirken kann, um letztlich gewährleisten zu können, diesen Kunden an das Unternehmen oder die Institution zu binden.
Customer Journey Maps lassen für die verschiedensten zwecke einsetzen, so beispielsweise auch für Bibliotheken wie man anhand der Studie von \cite[]{Marquez2015} deutlich sehen kann.
\newline
\newline
\textbf{User Experience}
User Experience (UX) beschreibt, ähnlich wie Customer Journey, alle eindrücke die ein Nutzer im Zusammenhang mit einem Unternehmen, Produkt oder Dienstleistung.
Immer mehr wird der begriff in der Mensch Computer Interaktion verwendet und besteht aus mehreren Komponenten. User Interface und Usability sind dabei die Komponenten, die 
den größten Einfluss haben und auch bei der Entwicklung den Nutzer in den Mittelpunkt stellen. Laut \cite[]{Norman2020} ist das wichtigste für eine gute User Experience, das
Verständnis für Nutzer und dessen Bedürfnisse.
\newline
\newline
\textbf{Vergleich}
\newline
Vergleicht man beide Prozesse und Publikationen in den jeweiligen Bereichen, bemerkt man, das sowohl die Definitionen sehr ähnlich sind, als auch das der Begriff "User" und "Customer"
bei beiden Prozessen Substitutiv verwendet wird. Daher gelangen wir, zumindest im Bereich der Customer Journey Map, zu dem Schluss, das ein Customer und ein User keiner
getrennten Betrachtung bedarf.



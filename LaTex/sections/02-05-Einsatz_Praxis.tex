\subsection{Vorteile von Customer Journey Maps}
Die Vorteile von Customer Journey Maps für bestehende, aber auch für neue Produkte sind zahlreich. 
Natürlich ist bei der Erstellung einer Customer Journey Map der User immer im Zentrum des Prozesses. Trotzdem ergeben sich auch für Unternehmen bzw. Anwender Vorteile.
Zwar sind Unternehmen nicht immer die Anwender von Customer Journey Maps, da diese sich auch zum Beispiel für Foschungszwecke hervorragend eignen, aber
Vorteile für nicht gewinnorientierte Anwender und Unternehmen unterscheiden sich nur in Feinheiten. Um unnötige Komplexität zu vermeiden wird im Folgenden
über Vorteile für Unternehmen gesprochen, diese lassen sich jedoch auch teilweise auf andere Anwender übertragen. 



\subsubsection{Vorteile für User durch die Nutzung von CJMs}
Für User ergeben sich im Kontext der Nutzung von Customer Journey Maps einige nicht zu vernachlässigende Vorteile. Gerade die starke Einflussmöglichkeit
des Users auf das zukünftige Produkt ist hier ein wichtiger Punkt. Durch den stark Nutzerzentrierten Designaspekt von Customer Journey Maps werden die Emotionen
und Ansichten von diesem zum wichtigsten Aspekt der Evaluierung des Produktes. Vorteilhaft ist dies beispielsweise, da Usern nun die Möglichkeit geboten wird
ein Produkt, welches sie sowieso schon verwenden, hin zu ihren Bedürfnissen zu verändern. Dies spart Usern die mühevolle Migration hin zu einem neuen Produkt,
welches eventuelle Bedürfnisse dann besser abbildet. Ein weiterer großer Pluspunkt, der vor allem durch die Nutzung von ethnografischen Methoden rührt, ist der,
dass auch individuelle Ansichten % TODO: (Anderes Wort)
des Users sehr gut abgebildet werden können. Subjektive Wahrnehmung, die etwa in ein User-Diary oder ein Semi-Strukturierten Interview gut integrierbar ist,
wird durch herkömmliche Datenanalyse vernachlässigt. Dies führt dazu, dass auch einzelne Meinungen von Usern nicht untergehen und zu wertvollen Ideen oder Anstößen
für Unternehmen heranwachsen können. Zusätzlich entspringen einer User fokusierten Designmethode natürlich auch Prozesse, welche stark an den User angepasst und damit
verhältnismäßig einfach für den User zu durchlaufen sind. Dies ergibt sich vor allem durch die Aufnahme der Emotionen von Usern in die Erstellung. Diese können helfen
unangenehme und bedrängende Situationen für den User zu vermeiden und diese elegant zu umgehen. 

\subsubsection{Vorteile für Unternehmen durch die Nutzung von CJMs}
Trotz des User fokusierten Designansruchs ergeben sich bei der Nutzung von Customer Journey Maps auch Vorteile für Unternehmen.  
Ein Aspekt der hierbei gleichermaßen für User, aber auch für Anwender einer CJM wichtig ist die User-Getriebenheit des Designprozesses.
Hier steht der User einer Dienstleistung wirklich im Vordergrund. Der Prozess zur Erstellung einer CJM sorgt hierbei auch %TODO Anders schreiben
dafür, dass diese user-zentrierte Arbeitsweise niemals in der Hintergrund gerät. Weiterhin ist die flexible Einsetzbarkeit von Customer
Journey Maps für Unternehmen sehr positiv. Mithilfe einer CJM lässt sich beispielsweise nicht nur der Werbeprozess für ein Produkt oder einen Prozess abbilden
 - Der Pre-Service Prozess - sondern auch der tatsächliche Nutzungsprozess dieses. So lassen sich sehr langwierige Prozessketten kompakt visualisieren. 
Diese Visualisierung unterstützt im Anschluss wiederum das unternehmensinterne Verständnis für das Produkt-Ökosystem.
Dieses kann nämlich gerade bei größeren Unternehmen sehr komplex und undurchschaubar werden.
% TODO: Hier vllt Beispiel von großem Ökosystem (MS-Azure, Apple ...)
Gerade in solchen Szenarien glänzen CJMs durch ihre vorteilhafte Darstellung.
Ein weiterer Vorteil der sich bei der Nutzung von CJMs für Unternehmen auftut ist die Einbeziehung der Emotionen von Usern in den Designprozess. 
Oft wird die Verarbeitung von Emotionen in der Datenanalyse außen vor gelassen, allerdings entgleitet damit auch die Chance neue,
bisher unentdeckete Lösungswege bzw. wichtige Berührungspunkte mit dem Produkt oder Prozess zu entdecken.
Dies geht daher über die übliche quantitative Analyse der vorhandenen Daten hinaus und bietet somit, gerade auch durch die tiefgreifende Integration 
ethnografischer Methoden, die Möglichkeit die Perspektive des Users einzunehmen. 
% TODO: Vllt. Noch Vorteile?
